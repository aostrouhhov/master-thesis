% !TeX spellcheck = ru_RU
% !TEX root = vkr.tex

В рамках данной работы были достигнуты следующие результаты.
\begin{itemize}
  \item Проведён анализ существующих в разных ОС реализаций слоя двоичной совместимости с GNU/Linux. Выделены два подхода: эмуляция и паравиртуализация. Сделан выбор в пользу метода паравиртуализации ядра Linux.
  \item Разработана архитектура подсистемы двоичной совместимости с GNU/Linux, которая может быть реализована в ряде разных ОС. Это достигается за счёт использования библиотеки LKL в качестве паравиртуализированного ядра.
  \item На основе разработанной архитектуры реализована и настроена подсистема для ОС Embox.
  \item Проведена апробация созданной подсистемы. В Embox запущены демонстрационные приложения \texttt{linux\_cat} и \texttt{linux\_ls}. Продемонстрирована работа с файловой системой \texttt{procfs}.
\end{itemize}

Логичным продолжением данной работы будет являться исследование возможности поддержки слоем совместимости разделяемых библиотек и запуска двоичных файлов, скомпонованных динамически. В частности, поддержка библиотеки GLIBC (как разделяемой) сильно увеличит область применимости предлагаемого слоя двоичной совместимости.
