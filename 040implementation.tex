% !TeX spellcheck = ru_RU
% !TEX root = vkr.tex

На основе описанной архитектуры для ОС Embox был создан слой двоичной совместимости с приложениями GNU/Linux. Была реализована рассматриваемая в предыдущей главе подсистема, использующая в качестве паравиртуализируемого ядра Linux библиотеку LKL. Подсистема содержит все необходимые составляющие, включая функцию-обработчик системных вызовов Linux. К изменениям, затрагивающим концептуальные части исходного кода ОС Embox, можно отнести дополнение структуры задачи Embox полями для хранения ключа TLS и информации о разрешении на исполнение системных вызовов Linux. Автором был сформирован запрос на привнесение изменений в проект Embox~\cite{pull-request}.

Далее представлены некоторые детали реализации слоя двоичной совместимости с GNU/Linux, специфичные для ОС Embox. Описание некоторых их них может быть полезным при реализации слоя совместимости в других ОС.

% ----

\subsection{Модуль подсистемы Linux в ОС Embox}

В Embox все компоненты ОС существуют в виде модулей. Собираемый образ ОС будет содержать только те модули, которые были включены в файл конфигурации компонентов. Ранее библиотека LKL была внедрена в Embox в виде модуля~\cite{lkl-in-embox-patch}. Этот модуль был использован для реализации внутри ОС Embox упомянутой в разделе \ref{chapter-with-subsystem} подсистемы, осуществляющей инициализацию, настройку и проверку библиотеки LKL, а также реализующей функцию-обработчик системных вызовов Linux.

% ----

\subsection{Запуск сторонних исполняемых файлов}

Все программы и библиотеки в ОС Embox включаются в образ ОС на этапе сборки. Для запуска сторонних исполняемых файлов используется инструмент-загрузчик \texttt{load\_app}.

В структуру, отвечающую за представление задачи в ОС Embox, было добавлено поле для идентификации процесса как исполняющего двоичный код приложения Linux. Инструмент \texttt{load\_app} был модифицирован для идентификации процесса как исполняющего Linux приложение. Все дочерние процессы автоматически наследуют значение этого поля от родительского процесса при создании. Таким образом, пользователь разрешает исполнение системных вызовов Linux только тем процессам, которые создаются при вызове приложений Linux через инструмент \texttt{load\_app}.

% ----

\subsection{Вывод текста из LKL в терминал ОС Embox}

Часть взаимодействия между LKL и Embox происходит при помощи \textit{виртуальных блочных устройств} LKL. Благодаря этому механизму возможны такие действия, как вывод текста из LKL в терминал Embox, или взаимодействие Linux приложения с файловой системой Embox.

В подсистеме слоя совместимости было реализовано такое блочное устройство, которое выводит текст из LKL в терминал ОС Embox. Изначально LKL не имеет доступа к терминалу Embox. Для организации этого доступа в подсистеме с LKL создается \textit{виртуальное блочное устройство}, являющееся структурой \texttt{lkl\_disk}. Для этого блочного устройства определяются функции обработки запросов на чтение и запись. Далее, за блочным устройством закрепляется \textit{специальный блочный файл}, который имеет путь и имя \texttt{/vda}. Если системный вызов Linux \texttt{write} совершит запись в файловый дескриптор с номером 1 (стандартный вывод), то произойдёт запись информации в файл \texttt{/vda}. В \textit{обработчике запроса на запись} того устройства, на которое указывает файл \texttt{/vda}, реализован вывод текста в терминал ОС Embox (стандартными средствами Embox~--- функцией \texttt{printf}).

Подобных виртуальных устройств может быть произвольное количество, а обработчик каждого может обладать любым функционалом. Благодаря этому становится возможным, например, смонтировать в окружении подсистемы с ядром Linux файловую систему из ОС Embox для того, чтобы процесс мог работать с ней в контексте Linux.
